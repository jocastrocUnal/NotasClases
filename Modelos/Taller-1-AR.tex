% Options for packages loaded elsewhere
\PassOptionsToPackage{unicode}{hyperref}
\PassOptionsToPackage{hyphens}{url}
%
\documentclass[
]{article}
\usepackage{amsmath,amssymb}
\usepackage{lmodern}
\usepackage{iftex}
\ifPDFTeX
  \usepackage[T1]{fontenc}
  \usepackage[utf8]{inputenc}
  \usepackage{textcomp} % provide euro and other symbols
\else % if luatex or xetex
  \usepackage{unicode-math}
  \defaultfontfeatures{Scale=MatchLowercase}
  \defaultfontfeatures[\rmfamily]{Ligatures=TeX,Scale=1}
\fi
% Use upquote if available, for straight quotes in verbatim environments
\IfFileExists{upquote.sty}{\usepackage{upquote}}{}
\IfFileExists{microtype.sty}{% use microtype if available
  \usepackage[]{microtype}
  \UseMicrotypeSet[protrusion]{basicmath} % disable protrusion for tt fonts
}{}
\makeatletter
\@ifundefined{KOMAClassName}{% if non-KOMA class
  \IfFileExists{parskip.sty}{%
    \usepackage{parskip}
  }{% else
    \setlength{\parindent}{0pt}
    \setlength{\parskip}{6pt plus 2pt minus 1pt}}
}{% if KOMA class
  \KOMAoptions{parskip=half}}
\makeatother
\usepackage{xcolor}
\IfFileExists{xurl.sty}{\usepackage{xurl}}{} % add URL line breaks if available
\IfFileExists{bookmark.sty}{\usepackage{bookmark}}{\usepackage{hyperref}}
\hypersetup{
  pdftitle={Taller 1 de AR},
  pdfauthor={Joan Nicolas Castro Cortes},
  hidelinks,
  pdfcreator={LaTeX via pandoc}}
\urlstyle{same} % disable monospaced font for URLs
\usepackage[margin=1in]{geometry}
\usepackage{color}
\usepackage{fancyvrb}
\newcommand{\VerbBar}{|}
\newcommand{\VERB}{\Verb[commandchars=\\\{\}]}
\DefineVerbatimEnvironment{Highlighting}{Verbatim}{commandchars=\\\{\}}
% Add ',fontsize=\small' for more characters per line
\usepackage{framed}
\definecolor{shadecolor}{RGB}{248,248,248}
\newenvironment{Shaded}{\begin{snugshade}}{\end{snugshade}}
\newcommand{\AlertTok}[1]{\textcolor[rgb]{0.94,0.16,0.16}{#1}}
\newcommand{\AnnotationTok}[1]{\textcolor[rgb]{0.56,0.35,0.01}{\textbf{\textit{#1}}}}
\newcommand{\AttributeTok}[1]{\textcolor[rgb]{0.77,0.63,0.00}{#1}}
\newcommand{\BaseNTok}[1]{\textcolor[rgb]{0.00,0.00,0.81}{#1}}
\newcommand{\BuiltInTok}[1]{#1}
\newcommand{\CharTok}[1]{\textcolor[rgb]{0.31,0.60,0.02}{#1}}
\newcommand{\CommentTok}[1]{\textcolor[rgb]{0.56,0.35,0.01}{\textit{#1}}}
\newcommand{\CommentVarTok}[1]{\textcolor[rgb]{0.56,0.35,0.01}{\textbf{\textit{#1}}}}
\newcommand{\ConstantTok}[1]{\textcolor[rgb]{0.00,0.00,0.00}{#1}}
\newcommand{\ControlFlowTok}[1]{\textcolor[rgb]{0.13,0.29,0.53}{\textbf{#1}}}
\newcommand{\DataTypeTok}[1]{\textcolor[rgb]{0.13,0.29,0.53}{#1}}
\newcommand{\DecValTok}[1]{\textcolor[rgb]{0.00,0.00,0.81}{#1}}
\newcommand{\DocumentationTok}[1]{\textcolor[rgb]{0.56,0.35,0.01}{\textbf{\textit{#1}}}}
\newcommand{\ErrorTok}[1]{\textcolor[rgb]{0.64,0.00,0.00}{\textbf{#1}}}
\newcommand{\ExtensionTok}[1]{#1}
\newcommand{\FloatTok}[1]{\textcolor[rgb]{0.00,0.00,0.81}{#1}}
\newcommand{\FunctionTok}[1]{\textcolor[rgb]{0.00,0.00,0.00}{#1}}
\newcommand{\ImportTok}[1]{#1}
\newcommand{\InformationTok}[1]{\textcolor[rgb]{0.56,0.35,0.01}{\textbf{\textit{#1}}}}
\newcommand{\KeywordTok}[1]{\textcolor[rgb]{0.13,0.29,0.53}{\textbf{#1}}}
\newcommand{\NormalTok}[1]{#1}
\newcommand{\OperatorTok}[1]{\textcolor[rgb]{0.81,0.36,0.00}{\textbf{#1}}}
\newcommand{\OtherTok}[1]{\textcolor[rgb]{0.56,0.35,0.01}{#1}}
\newcommand{\PreprocessorTok}[1]{\textcolor[rgb]{0.56,0.35,0.01}{\textit{#1}}}
\newcommand{\RegionMarkerTok}[1]{#1}
\newcommand{\SpecialCharTok}[1]{\textcolor[rgb]{0.00,0.00,0.00}{#1}}
\newcommand{\SpecialStringTok}[1]{\textcolor[rgb]{0.31,0.60,0.02}{#1}}
\newcommand{\StringTok}[1]{\textcolor[rgb]{0.31,0.60,0.02}{#1}}
\newcommand{\VariableTok}[1]{\textcolor[rgb]{0.00,0.00,0.00}{#1}}
\newcommand{\VerbatimStringTok}[1]{\textcolor[rgb]{0.31,0.60,0.02}{#1}}
\newcommand{\WarningTok}[1]{\textcolor[rgb]{0.56,0.35,0.01}{\textbf{\textit{#1}}}}
\usepackage{longtable,booktabs,array}
\usepackage{calc} % for calculating minipage widths
% Correct order of tables after \paragraph or \subparagraph
\usepackage{etoolbox}
\makeatletter
\patchcmd\longtable{\par}{\if@noskipsec\mbox{}\fi\par}{}{}
\makeatother
% Allow footnotes in longtable head/foot
\IfFileExists{footnotehyper.sty}{\usepackage{footnotehyper}}{\usepackage{footnote}}
\makesavenoteenv{longtable}
\usepackage{graphicx}
\makeatletter
\def\maxwidth{\ifdim\Gin@nat@width>\linewidth\linewidth\else\Gin@nat@width\fi}
\def\maxheight{\ifdim\Gin@nat@height>\textheight\textheight\else\Gin@nat@height\fi}
\makeatother
% Scale images if necessary, so that they will not overflow the page
% margins by default, and it is still possible to overwrite the defaults
% using explicit options in \includegraphics[width, height, ...]{}
\setkeys{Gin}{width=\maxwidth,height=\maxheight,keepaspectratio}
% Set default figure placement to htbp
\makeatletter
\def\fps@figure{htbp}
\makeatother
\setlength{\emergencystretch}{3em} % prevent overfull lines
\providecommand{\tightlist}{%
  \setlength{\itemsep}{0pt}\setlength{\parskip}{0pt}}
\setcounter{secnumdepth}{-\maxdimen} % remove section numbering
\ifLuaTeX
  \usepackage{selnolig}  % disable illegal ligatures
\fi

\title{Taller 1 de AR}
\author{Joan Nicolas Castro Cortes}
\date{2022-04-05}

\begin{document}
\maketitle

2.1 En la tabla B.1 del apéndice aparecen datos sobre el desempeño de
los \(26\) equipos de la Liga Nacional de Fútbol en 1976. Se cree que la
cantidad de yardas ganadas por tierra por los contrarios (\(x_8\)) tiene
un efecto sobre la cantidad de juegos que gana un equipo (\(y\)).

\begin{Shaded}
\begin{Highlighting}[]
\NormalTok{tableB1 }\OtherTok{\textless{}{-}} \FunctionTok{read.csv}\NormalTok{(}\StringTok{"C:/Users/nico9/Documents/Notebooks/Analisis de regresion/r/LinearModels/tableB1.csv"}\NormalTok{,}\AttributeTok{sep =} \StringTok{";"}\NormalTok{)}
\NormalTok{tableB1}
\end{Highlighting}
\end{Shaded}

\begin{verbatim}
##               Team  y x_.1. x_.2. x_.3. x_.4. x_.5. x_.6. x_.7. x_.8. x_.9.
## 1       Washington 10  2113  1985  38,9   647     4   868  59,7  2205  1917
## 2        Minnesota 11  2003  2855  38,8   613     3   615    55  2096  1575
## 3      New England 11  2957  1737  40,1   600    14   914  65,6  1847  2175
## 4          Oakland 13  2285  2905  41,6   453    -4   957  61,4  1903  2476
## 5       Pittsburgh 10  2971  1666  39,2   538    15   836  66,1  1457  1866
## 6        Baltimore 11  2309  2927  39,7   741     8   786    61  1848  2339
## 7      Los Angeles 10  2528  2341  38,1   654    12   754  66,1  1564  2092
## 8           Dallas 11  2147  2737    37   783    -1   761    58  1821  1909
## 9          Atlanta  4  1689  1414  42,1   476    -3   714    57  2577  2001
## 10         Buffalo  2  2566  1838  42,3   542    -1   797  58,9  2476  2254
## 11         Chicago  7  2363  1480  37,3   480    19   984  67,5  1984  2217
## 12      Cincinnati 10  2109  2191  39,5   519     6   700  57,2  1917  1758
## 13       Cleveland  9  2295  2229  37,4   536    -5  1037  58,8  1761  2032
## 14          Denver  9  1932  2204  35,1   714     3   986  58,6  1709  2025
## 15         Detroit  6  2213  2140  38,8   583     6   819  59,2  1901  1686
## 16       Green Bay  5  1722  1730  36,6   526   -19   791  54,4  2288  1835
## 17         Houston  5  1498  2072  35,3   593    -5   776  49,6  2072  1914
## 18     Kansas City  5  1873  2929  41,1   553    10   789  54,3  2861  2496
## 19           Miami  6  2118  2268  38,2   696     6   582  58,7  2411  2670
## 20     New Orleans  4  1775  1983  39,3   783     7   901  51,7  2289  2202
## 21 New York Giants  3  1904  1792  39,7   381    -9   734  61,9  2203  1988
## 22   New York Jets  3  1929  1606  39,7   688   -21   627  52,7  2592  2324
## 23    Philadelphia  4  2080  1492  35,5   688    -8   722  57,8  2053  2550
## 24       St. Louis 10  2301  2835  35,3   741     2   683  59,7  1979  2110
## 25       San Diego  6  2040  2416  38,7   500     0   576  54,9  2048  2628
## 26   San Francisco  8  2447  1638  39,9   571    -8   848  65,3  1786  1776
## 27         Seattle  2  1416  2649  37,4   563   -22   684  43,8  2876  2524
## 28       Tampa Bay  0  1503  1503  39,3   470    -9   875  53,5  2560  2241
\end{verbatim}

y: Games won (per 14 - game season)\\
x1: Rushing yards (season)\\
x2: Passing yards (season)\\
x3: Punting average (yards/punt)\\
x4: Field goal percentage (FGs made/FGs attempted 2season)\\
x5: Turnover differential (turnovers acquired -- turnovers lost)\\
x6: Penalty yards (season)\\
x7: Percent rushing (rushing plays/total plays)\\
x8: Opponents ' rushing yards (season)\\
x9: Opponents ' passing yards (season)

\begin{enumerate}
\def\labelenumi{\alph{enumi}.}
\tightlist
\item
  Ajustar un modelo de regresión lineal simple que relacione los juegos
  ganados, y, con las yardas ganadas por tierra por los contrarios,
  \(x_8\).
\end{enumerate}

\textbf{\emph{Solución:}}

\begin{Shaded}
\begin{Highlighting}[]
\NormalTok{x }\OtherTok{=}\NormalTok{ tableB1[,}\DecValTok{10}\NormalTok{]}
\NormalTok{y }\OtherTok{=}\NormalTok{ tableB1[,}\DecValTok{2}\NormalTok{]}
\NormalTok{meanx }\OtherTok{=} \FunctionTok{mean}\NormalTok{(x)}
\NormalTok{meany }\OtherTok{=} \FunctionTok{mean}\NormalTok{(y)}
\NormalTok{Sxy }\OtherTok{=} \FunctionTok{sum}\NormalTok{(((x}\SpecialCharTok{{-}}\NormalTok{meanx))}\SpecialCharTok{*}\NormalTok{y)}
\NormalTok{Sxx }\OtherTok{=} \FunctionTok{sum}\NormalTok{(((x}\SpecialCharTok{{-}}\NormalTok{meanx)}\SpecialCharTok{\^{}}\DecValTok{2}\NormalTok{))}
\NormalTok{beta1 }\OtherTok{=}\NormalTok{ Sxy}\SpecialCharTok{/}\NormalTok{Sxx}
\NormalTok{beta0 }\OtherTok{=}\NormalTok{ meany }\SpecialCharTok{{-}}\NormalTok{ beta1}\SpecialCharTok{*}\NormalTok{meanx}

\FunctionTok{plot}\NormalTok{(x,y)}
\FunctionTok{abline}\NormalTok{(}\AttributeTok{a=}\NormalTok{beta0,}\AttributeTok{b=}\NormalTok{beta1) }\CommentTok{\# Los parametros son el intercepto y la pendiente que calculamos}
\end{Highlighting}
\end{Shaded}

\includegraphics{Taller-1-AR_files/figure-latex/unnamed-chunk-2-1.pdf}

\begin{enumerate}
\def\labelenumi{\alph{enumi}.}
\setcounter{enumi}{1}
\tightlist
\item
  Formar la tabla de análisis de varianza y probar el significado de la
  regresión.
\end{enumerate}

\textbf{\emph{Solución:}}

\begin{Shaded}
\begin{Highlighting}[]
\NormalTok{xylm }\OtherTok{\textless{}{-}} \FunctionTok{lm}\NormalTok{(y }\SpecialCharTok{\textasciitilde{}}\NormalTok{ x)}
\FunctionTok{summary}\NormalTok{(xylm)}
\end{Highlighting}
\end{Shaded}

\begin{verbatim}
## 
## Call:
## lm(formula = y ~ x)
## 
## Residuals:
##    Min     1Q Median     3Q    Max 
## -3.804 -1.591 -0.647  2.032  4.580 
## 
## Coefficients:
##              Estimate Std. Error t value Pr(>|t|)    
## (Intercept) 21.788251   2.696233   8.081 1.46e-08 ***
## x           -0.007025   0.001260  -5.577 7.38e-06 ***
## ---
## Signif. codes:  0 '***' 0.001 '**' 0.01 '*' 0.05 '.' 0.1 ' ' 1
## 
## Residual standard error: 2.393 on 26 degrees of freedom
## Multiple R-squared:  0.5447, Adjusted R-squared:  0.5272 
## F-statistic:  31.1 on 1 and 26 DF,  p-value: 7.381e-06
\end{verbatim}

\begin{Shaded}
\begin{Highlighting}[]
\FunctionTok{anova}\NormalTok{(xylm)}
\end{Highlighting}
\end{Shaded}

\begin{verbatim}
## Analysis of Variance Table
## 
## Response: y
##           Df Sum Sq Mean Sq F value    Pr(>F)    
## x          1 178.09 178.092  31.103 7.381e-06 ***
## Residuals 26 148.87   5.726                      
## ---
## Signif. codes:  0 '***' 0.001 '**' 0.01 '*' 0.05 '.' 0.1 ' ' 1
\end{verbatim}

\begin{Shaded}
\begin{Highlighting}[]
\NormalTok{alpha }\OtherTok{=} \FloatTok{0.001}

\NormalTok{SS\_T }\OtherTok{=} \FunctionTok{sum}\NormalTok{((y}\SpecialCharTok{\^{}}\DecValTok{2}\NormalTok{))}\SpecialCharTok{{-}}\NormalTok{(((}\FunctionTok{sum}\NormalTok{(y))}\SpecialCharTok{\^{}}\DecValTok{2}\NormalTok{)}\SpecialCharTok{/}\FunctionTok{length}\NormalTok{(x))}
\NormalTok{SS\_Res }\OtherTok{=}\NormalTok{ SS\_T }\SpecialCharTok{{-}}\NormalTok{ (beta1}\SpecialCharTok{*}\NormalTok{Sxy)}
\NormalTok{MS\_Res }\OtherTok{=}\NormalTok{ SS\_Res}\SpecialCharTok{/}\NormalTok{(}\FunctionTok{length}\NormalTok{(x) }\SpecialCharTok{{-}} \DecValTok{2}\NormalTok{)}
\NormalTok{SS\_R }\OtherTok{=}\NormalTok{ beta1}\SpecialCharTok{*}\NormalTok{Sxy}
\NormalTok{df\_R }\OtherTok{=} \DecValTok{1}
\NormalTok{df\_Res }\OtherTok{=} \FunctionTok{length}\NormalTok{(x) }\SpecialCharTok{{-}} \DecValTok{2}
\NormalTok{df\_T }\OtherTok{=} \FunctionTok{length}\NormalTok{(x) }\SpecialCharTok{{-}}\DecValTok{1}
\NormalTok{MS\_R }\OtherTok{=}\NormalTok{ SS\_R}\SpecialCharTok{/}\NormalTok{df\_R}
\NormalTok{MS\_Res }\OtherTok{=}\NormalTok{ SS\_Res}\SpecialCharTok{/}\NormalTok{df\_Res}
\NormalTok{F0 }\OtherTok{=}\NormalTok{ MS\_R}\SpecialCharTok{/}\NormalTok{MS\_Res }
\NormalTok{F0test }\OtherTok{=} \FunctionTok{qf}\NormalTok{(}\DecValTok{1}\SpecialCharTok{{-}}\NormalTok{alpha,}\AttributeTok{df1 =} \DecValTok{1}\NormalTok{,}\AttributeTok{df2 =}\NormalTok{ (}\FunctionTok{length}\NormalTok{(x)}\SpecialCharTok{{-}}\DecValTok{2}\NormalTok{))}
\NormalTok{pvalue\_F0 }\OtherTok{=} \DecValTok{1}\SpecialCharTok{{-}}\FunctionTok{pf}\NormalTok{(F0,}\AttributeTok{df1 =} \DecValTok{1}\NormalTok{,}\AttributeTok{df2 =}\NormalTok{ (}\FunctionTok{length}\NormalTok{(x)}\SpecialCharTok{{-}}\DecValTok{2}\NormalTok{))}

\NormalTok{SS\_R}
\end{Highlighting}
\end{Shaded}

\begin{verbatim}
## [1] 178.0923
\end{verbatim}

\begin{Shaded}
\begin{Highlighting}[]
\NormalTok{SS\_Res}
\end{Highlighting}
\end{Shaded}

\begin{verbatim}
## [1] 148.872
\end{verbatim}

\begin{Shaded}
\begin{Highlighting}[]
\NormalTok{SS\_T}
\end{Highlighting}
\end{Shaded}

\begin{verbatim}
## [1] 326.9643
\end{verbatim}

\begin{Shaded}
\begin{Highlighting}[]
\NormalTok{df\_R}
\end{Highlighting}
\end{Shaded}

\begin{verbatim}
## [1] 1
\end{verbatim}

\begin{Shaded}
\begin{Highlighting}[]
\NormalTok{df\_Res}
\end{Highlighting}
\end{Shaded}

\begin{verbatim}
## [1] 26
\end{verbatim}

\begin{Shaded}
\begin{Highlighting}[]
\NormalTok{df\_T}
\end{Highlighting}
\end{Shaded}

\begin{verbatim}
## [1] 27
\end{verbatim}

\begin{Shaded}
\begin{Highlighting}[]
\NormalTok{MS\_R}
\end{Highlighting}
\end{Shaded}

\begin{verbatim}
## [1] 178.0923
\end{verbatim}

\begin{Shaded}
\begin{Highlighting}[]
\NormalTok{MS\_Res}
\end{Highlighting}
\end{Shaded}

\begin{verbatim}
## [1] 5.725845
\end{verbatim}

\begin{Shaded}
\begin{Highlighting}[]
\NormalTok{F0}
\end{Highlighting}
\end{Shaded}

\begin{verbatim}
## [1] 31.10324
\end{verbatim}

\begin{Shaded}
\begin{Highlighting}[]
\NormalTok{pvalue\_F0}
\end{Highlighting}
\end{Shaded}

\begin{verbatim}
## [1] 7.380709e-06
\end{verbatim}

\begin{longtable}[]{@{}
  >{\raggedright\arraybackslash}p{(\columnwidth - 8\tabcolsep) * \real{0.1739}}
  >{\raggedright\arraybackslash}p{(\columnwidth - 8\tabcolsep) * \real{0.1739}}
  >{\centering\arraybackslash}p{(\columnwidth - 8\tabcolsep) * \real{0.2174}}
  >{\centering\arraybackslash}p{(\columnwidth - 8\tabcolsep) * \real{0.2174}}
  >{\centering\arraybackslash}p{(\columnwidth - 8\tabcolsep) * \real{0.2174}}@{}}
\toprule
\begin{minipage}[b]{\linewidth}\raggedright
Source of Variation
\end{minipage} & \begin{minipage}[b]{\linewidth}\raggedright
Sum of Squares
\end{minipage} & \begin{minipage}[b]{\linewidth}\centering
Degrees of Freedom
\end{minipage} & \begin{minipage}[b]{\linewidth}\centering
Mean Square
\end{minipage} & \begin{minipage}[b]{\linewidth}\centering
F\_(0)
\end{minipage} \\
\midrule
\endhead
Regression & SS\_(R)= hat(beta)\emph{(1)S}(xy) & 1 & MS\_(R) &
MS\_(R)//MS\_(``Res'') \\
Residual & SS\_(``Res'')=SS\_(T)- hat(beta)\emph{(1)S}(xy) & n-2 &
MS\_(``Res'') & \\
Total & SS\_(T) & n-1 & & \\
\bottomrule
\end{longtable}

\begin{longtable}[]{@{}
  >{\raggedright\arraybackslash}p{(\columnwidth - 10\tabcolsep) * \real{0.1429}}
  >{\raggedright\arraybackslash}p{(\columnwidth - 10\tabcolsep) * \real{0.1429}}
  >{\centering\arraybackslash}p{(\columnwidth - 10\tabcolsep) * \real{0.1786}}
  >{\centering\arraybackslash}p{(\columnwidth - 10\tabcolsep) * \real{0.1786}}
  >{\centering\arraybackslash}p{(\columnwidth - 10\tabcolsep) * \real{0.1786}}
  >{\centering\arraybackslash}p{(\columnwidth - 10\tabcolsep) * \real{0.1786}}@{}}
\toprule
\begin{minipage}[b]{\linewidth}\raggedright
Source of Variation
\end{minipage} & \begin{minipage}[b]{\linewidth}\raggedright
Sum of Squares
\end{minipage} & \begin{minipage}[b]{\linewidth}\centering
Degrees of Freedom
\end{minipage} & \begin{minipage}[b]{\linewidth}\centering
Mean Square
\end{minipage} & \begin{minipage}[b]{\linewidth}\centering
\(F_0\)
\end{minipage} & \begin{minipage}[b]{\linewidth}\centering
\$P value
\end{minipage} \\
\midrule
\endhead
Regression & 178.0923 & 1 & 178.092 & 31.10324 & 7.380709e-06 \\
Residual & 148.872 & 26 & 5.725845 & & \\
Total & 326.9643 & 27 & & & \\
\bottomrule
\end{longtable}

\begin{enumerate}
\def\labelenumi{\alph{enumi}.}
\setcounter{enumi}{2}
\item
  Determinar un intervalo de confianza de 95\% para la pendiente.
\item
  ¿Qué porcentaje de variabilidad total da \(y\), y explica este modelo?
\item
  Determinar un intervalo de confianza de \(95\)\% para la cantidad
  promedio de juegos ganados, si la distancia ganada por tierra por los
  contrarios se limita a \(2000\) yardas.
\end{enumerate}

2.2 Supóngase que se quiere usar el modelo desarrollado en el problema
2.1 para pronosticar la cantidad de juegos que ganará un equipo si puede
limitar los avances por tierra de sus contrarios a 1 800 yardas.
Determinar un estimado de punto de la cantidad de juegos ganados
cuando.vg = I 800. Determinar un intervalo de predicción de 90\% para la
cantidad de juegos ganados. 2.3 La tabla B.2 del apéndice contiene datos
reunidos durante un proyecto de energía solar en el Tecnológico de
Georgia. a. Ajustar un modelo de regresión lineal simple que relacione
el flujo total de calory (kilowatts) con la deflexión radial de los
rayos desviados x4 (milirradianes). b. Formar la tabla de análisis de
varianza y probar la significancia de la regresión. c.~Determinar un
intervalo de confianza de 99\% para la pendiente. d.~Calcular R2. e.
Determinar un intervalo de confianza de 95\% para el flujo promedio de
calor, cuando la deflexión radial es 16.5 milirradianes. 2.4 La tabla
B.3 del apéndice contiene datos sobre el rendimiento de la gasolina, en
millas, de 32 automóviles diferentes. a. Ajustar un modelo de regresión
lineal simple que relacione el rendimiento de la gasolina y (millas por
galón) y la cilindrada del motor (pulgadas cúbicas). b. Formar la tabla
de análisis de varianza y prueba de significancia de la regresión.
c.~¿Qué porcentaje de la variabilidad total del rendimiento de la
gasolina explica la relación lineal con la cilindrada del motor?
d.~Determinar un intervalo de confianza de 95\% para el rendimiento
promedio de gasolina, si el desplazamiento del motor es 275 pulg3. e.
Suponer que se desea pronosticar el rendimiento de gasolina que tiene un
coche con motor de 275 pulg3. Determine un estimado puntual para el
rendimiento. Determinar un intervalo de predicción de 95\% para el
rendimiento. .f Comparar los dos intervalos obtenidos en las partes d y
e. Explicar la diferencia entre ellos. ¿Cuál es más amplio y por
qué?REGRESIóN LINEAL SIMPLE 55 2.5 Acerca de los datos sobre rendimiento
de gasolina en la tabla B.3 del apéndice, repetir el problema 2.4
(partes a, b y c) usando el peso del vehículo, xI0, como la variable
regresora. Con base en una comparación entre los dos modelos, ¿se puede
llegar a la conclusión de que x \{ es mejor opción como regresor que
x10? La tabla B.4 del apéndice presenta datos de 27 casas vendidas en
Erie, Pennsylvania. a. Ajustar un modelo de regresión lineal simple que
relacione el precio de venta de la casa con los impuestos actuales
(JC,). b. Probar la significancia de la regresión. c.~¿Qué porcentaje de
la variabilidad total del precio de venta queda explicado con este
modelo? d.~Determinar un intervalo de confianza de 95\% para /3(. e.
Determinar un intervalo de confianza de 95\% para el precio promedio de
venta de una casa, para la cual los impuestos actuales son \$750. Se
cree que la pureza del oxígeno producido con un proceso de
fraccionamiento está relacionada con el porcentaje de hidrocarburos en
el condensador principal de la unidad de procesamiento. A continuación
se muestran los datos de veinte muestras. 2.6 2.7 Pureza (\%)
Hidrocarburos (\%) Pureza (\%) Hidrocarburos (\%) 86.91 1.02 96.73 99.42
98.66 1.46 89.85 1.11 1.55 90.28 86.34 92.58 1.43 1.55 1.11 96.07 1.55
1.01 93.65 1.40 87.33 0.95 87.31 1.15 86.29 91.86 1.11 95.00 96.85 85.20
90.56 1.01 0.87 0.99 95.61 1.43 0.95 89.86 1.02 0.98 a. Ajustar un
modelo de regresión lineal simple a los datos. b. Probar la hipótesis
H0: /3, = 0. c.~Calcular /?2. d.~Determinar un intervalo de confianza de
95\% para la pendiente. e. Determinar un intervalo de confianza de 95\%
para la pureza media, cuando el porcentaje de hidrocarburos es 1.00.
Para los datos de la planta de oxígeno en el problema 2.7, suponer que
la pureza y el porcentaje de hidrocarburos son variables aleatorias con
distribución normal conjunta. a. ¿Cuál es la correlación entre la pureza
del oxígeno y el porcentaje de hidrocarburos ? b. Probar la hipótesis
que p = 0. c.~Establecer un intervalo de confianza de 95\% para p. Con
los datos de tiempo de entrega de bebidas gaseosas de la tabla 2.9,
después de examinar el modelo original de regresión (Ej. 2.9), un
analista afirmó que el modelo es inválido, porque la ordenada al origen
no es cero. Dijo que si se entregaran cero cajas, el tiempo para
abastecer y dar servicio a la máquina sería cero, y por eso el modelo de
línea recta debe pasar por el origen. ¿Qué se le contestaría? Ajustar un
modelo sin ordenada al origen a esos datos, y determinar qué modelo es
el mejor. 2.8 2.956 INTRODUCCIóN AL ANáLISIS DE REGRESIóN LINEAL 2.10 A
continuación se muestran el peso y la presión sistólica sanguínea de 26
hombres seleccionados al azar, en el grupo de edades de 25 a 30. Suponer
que el peso y la presión sanguínea (BP) tienen distribución normal
conjunta. a. Determine una recta de regresión que relacione la presión
sistólica sanguínea con el peso. b. Estimar el coeficiente de
correlación. c.~Probar la hipótesis que p = 0. d.~Probar la hipótesis
que p = 0.6. e. Determinar un coeficiente de confianza de 95\% para p.
Persona Peso BP sistólica Persona Peso BP sistólica 1 165 130 14 172 153
2 167 133 15 159 128 3 180 150 16 168 132 4 155 128 17 174 149 5 212 151
18 183 158 6 175 146 19 215 150 7 190 150 20 195 163 8 210 140 21 180
156 9 200 148 22 143 124 10 149 125 23 240 170 11 158 133 24 235 165 12
169 135 25 192 160 13 170 150 26 187 159 2.11 Para los datos de peso y
presión sanguínea del problema 2.10, ajustar a ellos un modelo sin
ordenada al origen y compararlo con el obtenido en el problema 2.10.
¿Cuál modelo se diría que es el mejor? 2.12 Se cree que la cantidad de
libras de vapor usadas en una planta por mes está relacionada con la
temperatura ambiente promedio. A continuación se presentan los consumos
y las temperaturas del último año. Mes Temperatura Uso / 1000 Mes
Temperatura Uso / 1000 Ene. Feb. 21 185.79 214.47 288.03 424.84 454.68
539.03 Jul.~68 621.55 675.06 562.03 452.93 369.95 273.98 24 Ago. 74
Mar.~32 Sep. Oct. Nov. Die. 62 Abr. Mayo Jun. 47 50 50 41 59 30 a.
Ajustar un modelo de regresión lineal simple a los datos. b. Probar la
significancia de la regresión. c.~En la administración de la planta se
cree que un aumento de 1 grado en la temperatura ambiente promedio hace
aumentar 10 000 libras el consumo mensual de vapor. ¿Estos datos
respaldan la afirmación? d.~Determinar un intervalo de predicción de
99\% para el uso de vapor en un mes con temperatura ambiente promedio de
58°.REGRESIóN LINEAL SIMPLE 57 2.13 Davidson (``Update on Ozone Trends
in California's South Coast Air Basin'', Air and Waste, 43, 226, 1993)
estudió las concentraciones de ozono en la cuenca aérea de la costa sur
de California, durante los años 1976 a 1991. Cree que la cantidad de
días en que las concentraciones de ozono fueron mayores que 0.20 ppm (la
respuesta) depende del índice meteorológico estacional, que es el
promedio estacional de la temperatura con 850 milibars (el regresor). La
siguiente tabla muestra los datos. Año Días índice 1976 91 16.7 1977 105
17.1 1978 106 18.2 1979 108 18.1 1980 88 17.2 1981 91 18.2 1982 58 16.0
1983 82 17.2 1984 81 18.0 1985 65 17.2 1986 61 16.9 1987 48 17.1 1988 61
18.2 1989 43 17.3 1990 33 17.5 1991 36 16.6 a. Trazar un diagrama de
dispersión con los datos. b. Estimar la ecuación de predicción.
c.~Probar la significancia de la regresión. d.~Calcular y graficar las
bandas de 95\% de confianza y de predicción. 2.14 Hsuie, Ma y Tsai
(``Separación y caracterización de copoliésteres termotrópicos del ácido
p-hidroxibenzoico, ácido sebácico e hidroquinona'', Journal of Applied
Polymer Science, 56, 471-476, 1995) estudian el efecto de la relación
molar del ácido sebácico (el regresor) sobre la viscosidad intrínseca de
los copoliésteres (la respuesta). La siguiente tabla muestra los datos.
Radio Viscosidad 1.0 0.45 0.9 0.20 0.8 0.34 0.7 0.58 0.6 0.70 0.5 0.57
0.4 0.55 0.3 0.44 a. Trazar un diagrama de dispersión de los datos. b.
Estimar la ecuación de predicción. c.~Hacer un análisis completo y
adecuado (pruebas estadísticas, cálculo de R2,etcétera). d.~Calcular y
graficar las bandas de 95\% de confianza y de predicción.58 INTRODUCCIóN
AL ANáLISIS DE REGRESIóN LINEAL 2.15 Byers y Williams (``Viscosities of
Binary and Ternary Mixtures of Polynomatic Hydrocarbons'', Journal of
Chemical and Engineering Data, 32, 349-354, 1987) estudiaron el impacto
de la temperatura sobre la viscosidad de mezclas de tolueno y tetralina.
La tabla siguiente muestra los datos para mezclas con fracción molar de
tolueno igual a 0.4. Viscosidad (mPa •s) Temperatura (°C) 24.9 1.1330
0.9772 0.8532 0.7550 0.6723 0.6021 0.5420 0.5074 35.0 44.9 55.1 65.2
75.2 85.2 95.2 a. Estimar la ecuación de predicción. b. Hacer un
análisis completo del modelo. c.~Calcular y graficar las bandas de 95\%
de confianza y de predicción. 2.16 Carroll y Spiegelman (``The Effects
of Ignoring Small Measurement Errors in Precision Instrument
Calibration'', Journal of Quality Technology, 18, 170-173, 1986)
examinan la relación entre la presión en un tanque y el volumen de
líquido. La siguiente tabla muestra los datos. Use un paquete adecuado
de programas estadísticos para efectuar un análisis de ellos. Comente
los resultados obtenidos con la rutina del programa. Volumen Presión
Volumen Presión Volumen Presión 2084 4599 2842 6380 3789 8599 2084 4600
3030 6818 3789 8600 2273 5044 3031 6817 3979 9048 2273 5043 3031 6818
3979 9048 2273 5044 3221 7266 4167 9484 2463 5488 3221 7268 4168 9487
2463 5487 3409 7709 4168 9487 2651 5931 3410 7710 4358 9936 2652 5932
3600 8156 4358 9938 10377 10379 2652 5932 3600 8156 4546 2842 6380 3788
8597 4547 2.17 Para el modelo de regresión lineal simple y = 50 + 10.x +
£, donde e tiene NID (0, 16), suponer que se usan n = 20 pares de
observaciones para ajustar este modelo. Generar 500 muestras de 20
observaciones, tomando una observación para cada valor de x = 1, 1.5, 2,
. . . , 10 para cada muestra. a. Para cada muestra, calcular los
estimados de la pendiente y la ordenada al origen por mtar ínimos la
forma cuadrados de esos.histogramas Trazar histogramas . de los valores
muéstrales de\^{} y /3,. Comenb. Para cada muestra, calcular un estimado
de E(y\x = 5). Trazar un histograma de los estimados obtenidos. Comentar
la forma del histograma.REGRESIóN LINEAL SIMPLE 59 c.~Determinar un
intervalo de confianza de 95\% para la pendiente en cada muestra.
¿Cuántos de los intervalos contienen el valor verdadero /3, = 10? ¿Es lo
que se esperaba? d.~Para cada estimado de E(ylx = 5) en la parte b,
calcular el intervalo de confianza de 95\%. ¿Cuántos de esos intervalos
contienen el valor verdadero de E(y\x = 5) = 100? ¿Es lo que se
esperaba? 2.18 Repetir el problema 2.17 usando sólo 10 observaciones
para cada muestra y tomando una observación de cada nivel \emph{= 1, 2,
3, . . . , 10. ¿Qué impacto tiene usar n = 10 sobre las respuestas en el
problema 2.17? Comparar las longitudes de los intervalos de confianza y
el aspecto de los histogramas. 2.19 Se tiene el modelo de regresión
lineal simple y = /30 + /3,} + e, con E(e) = 0, Var íe) = cr 2 y eno
correlacionado. a. Demostrar que Cov(/3y, =-x\textless r2/Sxr b.
Demostrar que Cov(y, /3\textbar) = 0. 2.20 Se tiene el modelo de
regresión lineal simple y = /3,, + /3,\emph{+ e, con E(é) = 0, Var(e) =
\textless r 2, y e no correlacionado. a. Demostrar que E\{MSR )= a2 +
/32S„. b. Demostrar que E(MSRe¡) = a2. 2.21 Suponer que se debe ajustar
el modelo rectilíneo de regresión y = /3() + /3,},, pero que la
respuesta está afectada por una segunda variable x2, de tal modo que la
verdadera función de regresión es E(y) = A) + P*i + P2 X2 a. ¿Es
insesgado el estimador de mínimos cuadrados para la pendiente, en el
modelo original de regresión lineal simple? b. Demostrar el sesgo en
/3i- 2.22 Considérese el estimador la 2 de máxima verosimilitud de
\textless x 2 en el modelo de regresión lineal simple. Se sabe que a2 es
un estimador sesgado de cr 2. a. Demostrar la cantidad de sesgo en a2 .
b. ¿Qué sucede con el sesgo a medida que se hace grande el tamaño n de
la muestra? 2.23 Suponer que se debe ajustar una recta, y que se desea
que el error estándar de la pendiente sea el mínimo posible. Suponer que
la ``región de interés de x es-1 \textless{}\emph{\textless{} 1. ¿Dónde
se deber ían hacer las observaciones x (, x2,. . . ,}„? Comente los
aspectos prácticos de este plan de recolección de datos. 2.24 Se tienen
los datos del problema 2.12. Supóngase que el consumo de Vapor y la
temperatura ambiente tienen distribución normal conjunta. a. Determinar
la correlación entre el consumo de vapor y la temperatura ambiente
promedio mensual. b. Probar la hipótesis que p = 0. c.~Probar la
hipótesis que p = 0.5. d.~Determinar un intervalo de confianza de 99\%
para p. 2.25 Demostrar que el valor máximo de R2 es menor que I si los
datos contienen observaciones de y diferentes, para el mismo valor de
x.60 INTRODUCCIóN AL ANáLISIS DE REGRESIóN LINEAL 2.26 Se tiene el
modelo de regresión lineal simple y = P0 + plx + e donde se conoce la
ordenada al origen /Jn. a. Determinar el estimador de /3, de mínimos
cuadrados para este modelo. ¿Parece razonable ese resultado? b. ¿Cuál es
la varianza de la pendiente (/5,) para el estimador de mínimos cuadrados
que se determinó en la parte a? c.~Determinar un intervalo de confianza
de 100(1- a) por ciento para /3,. ¿Es este intervalo más angosto que el
estimador para el caso en el que se desconocen tanto la pendiente como
la ordenada al origen? 2.27 Se tienen los residuales de mínimos
cuadrados e¡ = y¡- \% i-1, 2,. .., n, obtenidos con el modelo de
regresión lineal simple. Determinar la varianza Var(e,) de los
residuales. ¿Es constante la varianza de los residuales? Comentar por
qué.

\end{document}
